%%%%%%%%%%%%%%%%%%%%%%%%%%%%%%%%%%%%%%%%%%%%%%%%%%%%%%%%%%
%%
%%  PROJECT: report
%%
%%  Created by s060370 on 10-06-10.
%%  Copyright 2010 __MyCompanyName__. 
%%  
%%  All rights reserved.
%%
%%%%%%%%%%%%%%%%%%%%%%%%%%%%%%%%%%%%%%%%%%%%%%%%%%%%%%%%%%

\documentclass[12pt,a4paper]{article}

%%%%%%%%%%%%%%%%%%%%%%%%%%%%%%%%%%%%%%%%%%%%%%%%%%%%%%%%%%
%% 
%% MARK: Packages declarations
%%
%%%%%%%%%%%%%%%%%%%%%%%%%%%%%%%%%%%%%%%%%%%%%%%%%%%%%%%%%%

% Running Headers and footers
\usepackage{fancyhdr}

% Use colored links
\usepackage[pdftex,colorlinks]{hyperref}

% More symbols
\usepackage{amsmath}
\usepackage{amssymb}
\usepackage{latexsym}

% Surround parts of graphics with box
\usepackage{boxedminipage}

% Package for including code in the document
\usepackage{listings}

\usepackage{verbatim}

%%%%%%%%%%%%%%%%%%%%%%%%%%%%%%%%%%%%%%%%%%%%%%%%%%%%%%%%%%
%% 
%% MARK: Fonts declarations
%%
%%%%%%%%%%%%%%%%%%%%%%%%%%%%%%%%%%%%%%%%%%%%%%%%%%%%%%%%%%

\usepackage{euler,eucal}

%%%%%%%%%%%%%%%%%%%%%%%%%%%%%%%%%%%%%%%%%%%%%%%%%%%%%%%%%%
%% 
%% MARK: Input macro file
%% 
%%%%%%%%%%%%%%%%%%%%%%%%%%%%%%%%%%%%%%%%%%%%%%%%%%%%%%%%%%

%%
%%  MACROS FOR report
%%
%%  Created by TU/e on 20-05-10.
%%  Copyright 2010 __MyCompanyName__. All rights reserved.
%%

%%%%%%%%%%%%%%%%%%%%%%%%%%%%%%%%%%%%%%%%%%%%%%%%%%%%%%%%%%
%% 
%% MARK: Page Display
%%
%%%%%%%%%%%%%%%%%%%%%%%%%%%%%%%%%%%%%%%%%%%%%%%%%%%%%%%%%%

\parindent 0mm 
\parskip 0.5em plus 1pt

%%%%%%%%%%%%%%%%%%%%%%%%%%%%%%%%%%%%%%%%%%%%%%%%%%%%%%%%%%
%% 
%% MARK: Theorem Environments
%%
%%%%%%%%%%%%%%%%%%%%%%%%%%%%%%%%%%%%%%%%%%%%%%%%%%%%%%%%%%

\newenvironment{header}{\em }{}

\newtheorem{claim}{}

\newenvironment{proof}{
\noindent{\sc Proof.}}{\nolinebreak$\blacksquare$
}

\newcommand{\artlabel}[1]{{\sc #1.}}



%%%%%%%%%%%%%%%%%%%%%%%%%%%%%%%%%%%%%%%%%%%%%%%%%%%%%%%%%%
%% 
%% Begin Document
%% 
%%%%%%%%%%%%%%%%%%%%%%%%%%%%%%%%%%%%%%%%%%%%%%%%%%%%%%%%%%

\begin{document}
	
	%%%%%%%%%%%%%%%%%%%%%%%%%%%%%%%%%%%%%%%%%%%%%%%%%%%%%%%%%%
	%% 
	%% MARK: Title & Data
	%% 
	%%%%%%%%%%%%%%%%%%%%%%%%%%%%%%%%%%%%%%%%%%%%%%%%%%%%%%%%%%
	
	\title{Report for Edocol}
	
	\author{Edin Dudojevic (0608206) \\
          Etienne van Delden (0618959) \\
          Anson van Rooij (0596312)}
		
	
	%%%%%%%%%%%%%%%%%%%%%%%%%%%%%%%%%%%%%%%%%%%%%%%%%%%%%%%%%%
	%% 
	%% MARK: Abstract
	%% 
	%%%%%%%%%%%%%%%%%%%%%%%%%%%%%%%%%%%%%%%%%%%%%%%%%%%%%%%%%%
	
	\begin{abstract}
		In this report we introduce our Web-application, explain for who it's built,
    what it does and how we have made it and with what technologies. We also mention
    what changes have been made from our initial planning. 
	\end{abstract}
		
	\maketitle
	
	%%%%%%%%%%%%%%%%%%%%%%%%%%%%%%%%%%%%%%%%%%%%%%%%%%%%%%%%%%
	%% 
	%% MARK: Introduction
	%% 
	%%%%%%%%%%%%%%%%%%%%%%%%%%%%%%%%%%%%%%%%%%%%%%%%%%%%%%%%%%
	
	%\section*{Introduction}
	   
	
	
	%%%%%%%%%%%%%%%%%%%%%%%%%%%%%%%%%%%%%%%%%%%%%%%%%%%%%%%%%%
	%% 
	%% MARK: The sections
	%% 
	%%%%%%%%%%%%%%%%%%%%%%%%%%%%%%%%%%%%%%%%%%%%%%%%%%%%%%%%%%
	
	\section{About Edocol}
    Our project, the web application called ``Edocol'',
    is an education cooperation website for documentation. The two 
    main purposes are aiding students and teachers in
    \begin{enumerate}
      \item learning about (technical) topics of their interest
      \item creating new, high quality documentation in a collaborative manner
    \end{enumerate}
    The two mechanisms for this are be
    \begin{enumerate}
      \item[a] sharing and creating information and,
      \item[b] finding people with the same interests/courses or with required 
      expertise/position (for ex. possible tutors or instructors).
    \end{enumerate}

    \begin{description}
      \item[What Edocol is:] Our web-app occupies a place between Wikipedia-like 
      information sources and social or professional networking sites; 
      it enables sharing information and helps education, but not anonymously or 
      in one specific format, and it facilitates cooperation, but always with a 
      focus on creating/improving documents and furthering one's education.
      \item[What Edocol is not:] This app is not intended to be a complete portal, 
      like Studyweb or Sakai, nor to be as hierarchically organized. It's not an 
      official information source like OWinfo or TUe.nl. It's not an evaluation 
      or validation system like Peach, a place to share unfinished, internal work 
      like a Subversion server, a project management system like Projexy, or an 
      email client like Outlook.
      \item[Features:]
      \begin{enumerate}
        \item Personal accounts
        \item an RDF database of ``documents'' and users
        \item following users
%        \item adding tags to documents
%        \item searching 
      \end{enumerate}
    \end{description}
    
    \emph{Personal accounts:} these are used to keep track of who have access to
    Edocol and to prevent anyone from adding and editting documents. Only users
    of the system can create now documents and only the original creator of a 
    document can edit it.\\
    \emph{RDF database:} An rdf database is used to easily add additional functionality.
    For example, tagging could be added by making a module that handles the requests and
    adds new properties and fields to new entries. The new type can be declared by 
    making an \emph{(id, edocol:type, ``tag'')} entry, and extra data that needs
    to be stored can be used by adding the triple \emph{(id, edocol:``property-name'', value)}.\\
    \emph{Following users:} When you follow a user, you can quickly go to that users
    personal page and see the documents that person has made. This helps the sharing
    and creation of information.\\
    
  
    \subsection{Changes from the original design} 
      The idea and design of Edocol is still the same. Only our must-have features
      have been implemented, though work has been made on a commenting system and tags,
      we were not able to implement those on time.
  
  \section{Technical solution}
    For the creation of Edocol, we have used the following frameworks and technologies:
    \begin{itemize}
      \item CherryPy\cite{cherrypy} for the HTTP framework
      \item Mako\cite{mako} as a system for making templates
      \item RDFlib\cite{rdflib} as an rdf framework and database storage
      \item (SQLite3 for storage)
      \item All of the code for both front and back end is written in Python
    \end{itemize}
    
    Edocol is split into two parts; a front-end (\verb|server.py|) and a 
    back-end (\verb|database.py| with the \verb|documentmethods|, 
    \verb|usermethods| and \verb|tagmethods| modules). The front end 
    handles the web interface; dynamically create webpages with the content 
    delivered by the back-end. The back-end is used to create the rdf graph, 
    the query the database and to store it to file. SQLite3 was used to store the
    rdf database, but a threading problem was encountered for which no solution
    has been found yet. The rdf database is now stored in a file. 
    
    \subsection{Changes from the original design}
    Instead of using Redland\cite{redland} as the rdf framework, we are using RDFlib. Redland is
    designed for UNIX based systems. With no UNIX server available at our disposal,
    we had to look for a different framework that would on both UNIX and Windows
    based systems.  
  
  \section{Use of semantic data}
    Data is added to the rdf graph in one of two types: either a user or a document.
    Both these types have properties, these can be the same like ``name'' or different 
    like ``email address'' and ``content'', that can be set to a certain value. 
    This data is then added as a triple \emph{(type, property, value)}. When the 
    front-end needs info, the properties and the value are sent. The front-end 
    can then filter what is needed.
    
  \section{Discussion}
    Despite that our target audience is the educational branch, edocol is not bound
    to be used that. Anyone could use it for any type of documentation, like software
    development, finances and medical data.\\
    By using an rdf database (and with tagging), creating an RSS-feed from any
    data would be easy. A lot of rdf based websites are also using RSS feeds. A
    switch to the Redland framework would make it even easier, for one if it's features
    is built-in support for RSS feeds.\\ \\
    What have we learned from this project? We have learned how to design and work
    with an rdf database. It helped us see the benefits of using rdf but also see
    the state in which development around rdf is. We had a lot of problems with the
    initial setup of any rdf framework and later on with actually using it. There 
    are multiple projects that each has it's own unique feature-set, but each also
    has it's problems including: missing SPARQL libraries, bugs, missing documentation,
    no up to date documentation, runs only on UNIX based systems \ldots \\
    We also learned how to create a website with dynamic content. Using templates 
    and a separate backend gave a real separation between creating a lay-out for content
    and the actual content. \\
    Working with python was great! Python is an easy to learn language with a lot of built-in
    features. While we had trouble with setting up rdflib, we learned a great deal 
    about python and how it works (and on limits of our own operating systems).
    

	
	%%%%%%%%%%%%%%%%%%%%%%%%%%%%%%%%%%%%%%%%%%%%%%%%%%%%%%%%%%
	%% 
	%% MARK: Bibliography
	%% 
	%%%%%%%%%%%%%%%%%%%%%%%%%%%%%%%%%%%%%%%%%%%%%%%%%%%%%%%%%%
	
\bibliographystyle{plain}
\begin{thebibliography}{}
  \bibitem{cherrypy}
    http://www.cherrypy.org/
  \bibitem{mako}
    http://www.makotemplates.org/
  \bibitem{rdflib}
    http://www.rdflib.net/
  \bibitem{redland}
    http://www.librdf.org/
\end{thebibliography}
	
\end{document}